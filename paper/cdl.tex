\documentclass{article} % For LaTeX2e
\usepackage{nips13submit_e,times}
\usepackage{hyperref}
\usepackage[noabbrev]{cleveref}
\usepackage{url}
\usepackage{brian}
\usepackage{booktabs}
\usepackage{graphicx}
\usepackage{microtype}
\usepackage[sort]{natbib}
%\documentstyle[nips13submit_09,times,art10]{article} % For LaTeX 2.09


\title{Efficient convolutional dictionary learning}


\author{
David S.~Hippocampus\thanks{ Use footnote for providing further information
about author (webpage, alternative address)---\emph{not} for acknowledging
funding agencies.} \\
Department of Computer Science\\
Cranberry-Lemon University\\
Pittsburgh, PA 15213 \\
\texttt{hippo@cs.cranberry-lemon.edu}
}

% The \author macro works with any number of authors. There are two commands
% used to separate the names and addresses of multiple authors: \And and \AND.
%
% Using \And between authors leaves it to \LaTeX{} to determine where to break
% the lines. Using \AND forces a linebreak at that point. So, if \LaTeX{}
% puts 3 of 4 authors names on the first line, and the last on the second
% line, try using \AND instead of \And before the third author name.

%\nipsfinalcopy % Uncomment for camera-ready version

\begin{document}


\maketitle

\begin{abstract}
We make convolutional coding faster.
\end{abstract}

\section{Introduction}
\cite{mairal2010}
% TODO:   2013-05-24 18:26:56 by Brian McFee <brm2132@columbia.edu>
% why convolutional coding?
%   shift-invariance, robustness against windowing/offset effects
%   fixed codebooks tend to be overcomplete, learn the same basis at multiple shifts
%       which is wasteful
%   convolutional coding lets us use a smaller codebook
%   positions of activations can be semantically meaningful, eg in music transcription
%       cite klapuri2012


\subsection{Related work}

\subsection{Preliminaries}
$\F:\R^d\rightarrow\C^d$ will denote the discrete Fourier transform, and $*$ will denote
convolution.
Time-domain signals will be represented by lower-case letters, \eg, $x \in \R^d$, while
frequency-domain representations will be denoted by capitals, $\F\{x\} = X \in \C^d$.
For a vector $u$, $\diag(u)$ will denote the square, diagonal matrix $U$ such that 
$U_{ii} = u_i$.

\section{Convolutional coding}
\label{sec:convcode}
Given a set of $m$ codewords ${\{d_k\}}_{k=1}^m \subset \R^d$ and a signal $x\in\R^d$, 
we would like to find a set of activation functions ${\{a_k\}}_{k=1}^m\subset\R^d$ which
can be used to reconstruct $x$ by convolving with the codewords:
\begin{align}
x &\approx \sum_{k=1}^m d_k * a_k.\label{eq:convapprox}
\end{align}
Because this parameterization increases the dimensionality of the representation from $d$ 
to $md$, we also desire that each activation function $a_k$ be sparse.  Formally, this is 
achieved by solving the following optimization problem:
\begin{align}
a(x) &\defeq \argmin_{\{a_k\}} \frac{1}{2} \left\|x - \sum_{k=1}^m d_k * a_k\right\|^2 +
\lambda \sum_{k=1}^m \omega(a_k), \label{eq:sisc}
\end{align}
where $\omega$ is a convex, sparsity-promoting regularization function (\eg, the 
$\ell_1$-norm), and $\lambda > 0 $ is a parameter to balance regularization against 
reconstruction accuracy. 

Because convolution is a linear operation, \eqref{eq:sisc} is convex in $a_k$.  
However, direct optimization is difficult.\bn{MORE}

\subsection{Coding in the frequency domain}
By applying Parseval's theorem and the convolution theorem,\footnote{We assume circular
convolutions throughout this article. However, linear convolutions can be accommodated
in the usual way by appropriate padding and truncation.} we can equivalently express the 
reconstruction term of \eqref{eq:sisc} in the frequency domain:\footnote{To ease
presentation, we assume the symmetrically normalized form of the Fourier transform, 
so that $\|x\| = \|\F\{x\}\|$. In general, a normalization constant may be included, 
or absorbed into the regularization parameter $\lambda$.}
\begin{align}
\left\|x - \sum_k d_k * a_k\right\|^2 &= \left\|\F\left\{x - \sum_k d_k *
a_k\right\}\right\|^2
= \left\| X - \sum_k D_k \circ A_k\right\|^2,\label{eq:fourier}
\end{align}
where the last equality follows by linearity of the Fourier transform, and $\circ$
denotes the Hadamard product: ${[u\circ v]}_i \defeq u_i v_i$.

While the right-hand side of \eqref{eq:fourier} nearly resembles a standard
least-squares objective, it can be put into a more familiar form via the following
lemma.
\begin{lemma}
Let $\{D_1, D_2, \dots, D_m\}, \{A_1, A_2, \dots, A_m\} \subset \C^d$.  
Then there exist ${\widehat{D} \in \C^{d\times dm}}$ and ${\vec{A} \in \C^{dm}}$ such that 
${\widehat{D}\vec{A} = \sum_{k=1}^m D_k \circ A_k}$.\label{lemma:hadamard}
\end{lemma}
\begin{proof}
$D_k$ and $A_k$ can be rearranged as follows:
\begin{align*}
\widehat{D} &\defeq \left[\diag(D_1),\; \diag(D_2),\; \dots,\; \diag(D_m)\right] \in \C^{d\times dm}\\
\vec{A} &\defeq \left[A_1\trans, A_2\trans, \dots,
A_m\trans\right]\trans \in \C^{dm}\\
\Rightarrow \quad {\left[\widehat{D}\vec{A}\right]}_i &= \sum_\ell \widehat{D}_{i\ell}
\vec{A}_\ell = \sum_k {[D_k]}_i {[A_k]}_i = \sum_k
{[D_k \circ A_k]}_i.
\end{align*}
\end{proof}

\Cref{lemma:hadamard} allows us to interchange the Hadamard product in
\eqref{eq:fourier} with a matrix product, so that 
$\left\|x - \sum_k d_k * a_k\right\|^2 =  \left\|X - \widehat{D}\vec{A}\right\|^2$,
but we are left with a least-squares objective function parameterized by the complex-valued 
vector $\vec{A}$. To facilitate the use of standard optimization techniques to solve for
$\vec{A}$, we separate out the real and imaginary components as follows:
\begin{equation}
X \mapsto \left[\begin{array}{r}\Re X \\ \Im X\end{array}\right] \in \R^{2d},
\quad\quad \widehat{D} \mapsto \left[\begin{array}{rr}\Re \widehat{D} & - \Im \widehat{D}\\
\Im \widehat{D} & \Re \widehat{D}\end{array}\right] \in \R^{2d\times 2dm},
\quad\quad \vec{A} \mapsto \left[\begin{array}{r}\Re \vec{A} \\ \Im \vec{A}\end{array}\right]
\in \R^{2dm}.\label{eq:realized}
\end{equation}
With a slight abuse of notation, we can equivalently solve the original sparse
convolutional coding as a least-squares problem:
\begin{equation}
\min_{A \in \R^{2dm}} \|X - \widehat{D}A\|^2 + \Omega(A),\label{eq:realvalued}
\end{equation}
where $A_k$ corresponds to the sub-vector of $A$ associated with the $k$th codewode, and 
$\Omega$ is a frequency-domain parameterization of the regularization function $\omega$.
From an optimal $A$, each activation function $a_k$ can be 
recovered by the inverse transform of the combined real and imaginary components
corresponding to $A_k$.

\subsection{ADMM solver}
\label{sec:solver}
At a first glance, \eqref{eq:realvalued} may seem no easier to solve than the original
formulation of \eqref{eq:sisc}, especially since in many cases, $\Omega$ may be most
naturally parameterized in the time domain.  We therefore apply the alternating
direction method of multipliers (ADMM) technique to separate the regularization and
reconstruction terms~\cite{boyd2011}.  This leads to the following equivalent
formulation:
\begin{equation}
\min_{A, Z} \frac{1}{2}\left\|X - \widehat{D}A\right\|^2 + \Omega(Z)
\quad\quad \suchthat\quad A - Z = 0.\label{eq:admm}
\end{equation}

Forming the (scaled-form) augmented Lagrangian for \eqref{eq:admm} with scaling
parameter $\rho > 0 $ yields
\[
\Ell_\rho(A,Z,W) = \frac{1}{2}\left\|X - \widehat{D}A\right\|^2 + \Omega(Z) +
\frac{\rho}{2}\|A - Z + W\|^2.
\]
Optimizing \eqref{eq:admm} then consists of iterating the following three steps until
convergence:
\begin{enumerate}
\item $A^{t+1} \leftarrow \argmin_A \Ell_\rho(A, Z^t, W^t) = \argmin_A \frac{1}{2}
\left\|X - \widehat{D}A\right\|^2 + \frac{\rho}{2}\left\|A - Z^t + W^t\right\|^2$,
\item $Z^{t+1} \leftarrow \argmin_Z \Ell_\rho(A^{t+1}, Z, W^t) =
\prox_{\rho^{-1}\Omega}\left(A^{t+1} + W^t \right)$,
\item $W^{t+1} \leftarrow W^t + A^{t+1} - Z^{t+1}$.
\end{enumerate}

Just as in the ADMM formulation of LASSO~\cite[chapter 6]{boyd2011}, the $A$ update takes 
the form of a ridge regression.  Setting the gradient to zero yields the update
equation:
\begin{align}
\zero = \nabla_A \Ell_\rho(A, Z^t, W^t) &= -\widehat{D}\trans \left(X - \widehat{D}A\right)
+ \rho(A - Z^t + W^t)\notag\\
\Rightarrow \quad A^{t+1} &= {\left(\widehat{D}\trans \widehat{D} + \rho I\right)}^{-1}
\left(\widehat{D}\trans X + \rho \left(Z^t - W^t\right)\right).\label{eq:a-update}
\end{align}
Now, observe that the inverted matrix is the sum of a $2dm\times 2dm$ diagonal matrix 
$\rho I$ and product of low-rank matrices $(\widehat{D}\trans\widehat{D})$.  
Applying the matrix inversion lemma yields
\begin{equation}
{\left(\widehat{D}\trans \widehat{D} + \rho I\right)}^{-1} = \rho^{-1} I - \rho^{-2} \widehat{D}\trans
{\left(I + \rho^{-1} \widehat{D}\widehat{D}\trans \right)}^{-1}
\widehat{D}.\label{eq:inversion}
\end{equation}
This alternative formulation superficially allows us to invert a $2d\times 2d$
matrix rather than the original $2dm\times 2dm$ matrix.  However, to see the full
benefits of this formulation, we will require the following results.

\begin{lemma}
Let ${u, v \in \R^{d\times m}}$ be any matrices, and let ${\widehat{u} \defeq
[\diag(u_1),\dots,\diag(u_m)] \in \R^{d\times dm}}$ denote the 
matrix of diagonals formed from the columns of $u$ as in \cref{lemma:hadamard} 
(and analogously for $v$).  Then $\widehat{u}\widehat{v}\trans = I \circ (uv\trans)$.
\label{lemma:product}
\end{lemma}

\begin{theorem}
Let $\widehat{D} \in \R^{2d\times 2dm}$ be as defined in \eqref{eq:realized}.  Then
$\widehat{D}\widehat{D}\trans \in \R^{2d\times 2d}$ is diagonal.
\end{theorem}
\begin{proof}
Let $U$ and $V$ denote the blocks of $\widehat{D}$ corresponding to the
real and imaginary components.  Then,
\begin{align*}
\widehat{D}\widehat{D}\trans = \left[\begin{array}{rr}U & -V\\ V & U\end{array}\right]
\left[\begin{array}{rr} U\trans & V\trans\\ -V\trans & U\trans\end{array}\right]
&= \left[\begin{array}{rr} UU\trans + VV\trans & UV\trans - VU\trans \\ VU\trans -
UV\trans & UU\trans + VV\trans\end{array} \right]\\
&= \left[\begin{array}{rr}UU\trans + VV\trans & 0 \\ 0 & UU\trans + VV\trans\end{array}
\right],
\end{align*}
where the last equality follows by symmetry of $\widehat{D}\widehat{D}\trans$, so that
$UV\trans = VU\trans$.  By \cref{lemma:product}, both $UU\trans$ and $VV\trans$ are 
diagonal, as is their sum.  So $\widehat{D}\widehat{D}\trans$ is diagonal as well.
\end{proof}


\subsection{Z-step: time-domain regularization}


\section{Dictionary learning}

\section{Experiments}
\section{Conclusion}
\appendix
\section{Deferred proofs from \Cref{sec:solver}}

\bibliographystyle{plain}
\bibliography{refs}

\end{document}
